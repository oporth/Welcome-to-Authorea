\subsection{Do the flares originate from the inner knot?}
One feature which was of particular interest during the multi-wavelength campaigns was the inner knot of the Crab nebula. This feature is observed $\approx$0.6'' south east of the pulsar (see Fig ). It is detected in the infrared and optical bands. In radio and X-rays it has so far not be detected XXXX, as shown in Fig. It was known, that the inner knot is a dynamical feature, its distance to the pulsar can vary by and its brightness varies by. 

As will be presented in detail in section \ref{sec:knot}, the inner knot was one of the prime candidates for the site of origin of the gamma-ray flares. In fact, it was proposed by , that most of the emission above 100 MeV could come from this feature. Therefore particularly dens observations targeting the inner knot where performed by the Keck Observatory and HST. However, non of the knot properties was found to correlate to the gamma-ray emission by XXX. One example is shown in Fig, where the gamma-ray flux is shown as a function of the distance of the knot. While this feature is therefore likely not the site of origin of the gamma-ray flares,