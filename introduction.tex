\section{Introduction (RB, AL)}
At the end of their life massive stars leave behind some of the most powerful sources in the sky: neutron stars or black holes. The environment around these fascinating objects allows
us to test conditions which can often not be replicated on Earth. Among these systems, pulsar wind nebula (PWN) are of particular interest. In them, the relativistic plasma wind ejected by the rotating neutron star interacts with the ambient medium. The latter is mostly composed of the ejecta of the original star explosion. This interaction leads to bright synchrotron and inverse Compton (IC) emission of the wind electrons, producing some of the brightest sources in the sky (electrons and positrons are referred to together as electrons throughout this text).

Modeling the emission of PWN reveals that these systems are able to accelerate particles to very high energies of up to several PeV. The ultimate energy source of this acceleration is the electric potential induced by the rotation of the neutron star. As neutron stars are highly magnetized and conducting, they act as a unipolar inductor, creating an electric potential of between their poles and their equator. This electromagnetic energy is carried of of the pulsar magnetosphere with the pulsar wind. Where and how is is transformed into kinetic particle energy is a matter of intense research.

In comparison to other sources of relativistic plasma, PWN con be resolved in great detail, as shown in Fig. \ref{fig:rescomp}. Scales close to the gyroradius of the emitting electrons can be resolved by current radio, optical and X-ray telescopes. Particularly in two systems, the Crab and Vela PWN, the plasma flow can be resolved down to scales of $\approx 10$ light days. These systems provide perfect test beds to study the behaviours of relativistic magnetized plasma, which is thought to be present in other high energy sources as Gamma-Ray Bursts or Active Galactic Nuclei \citep{Berger_2014,Massaro_2015}. The discovery of gamma-ray flares from the Crab PWN revealed very efficient and rapid acceleration of particles in this system. Strong and rapid gamma-ray flares are also observed in GRBs and AGN (see e.g. \citet{Ackermann_2013} and \citet{Aharonian_2009}). It is likely, that common mechanisms as magnetic reconnection or shock acceleration are responsible for the acceleration in all of these systems (see \citet{Kagan_2015} and \citet{Sironi_2015} for recent reviews).

In contrast to the inner region of the Vela PWN, the Crab PWN has been detected across the electromagnetic spectrum. The plasma motion is resolved in space and time with unmatched resolution across the electromagnetic spectrum in this system (see \citet{Hester_2008} and \citet{BuehlerBlandford2013a} for reviews). It has therefore been the target of most theoretical studies of PWN. In particular, relativistic magntohydrodynamic (RMHD) simulations have provided deep insides into the plasma behaviour over the past decades (see e.g.  \citet{Komissarov_2004,Del_Zanna_2006,Porth_2013}). Qualitatively, the global plasma flow in these systems is thought to be understood to date. The energy of the wind emitted by the pulsar is concentrated towards the equator. The wind is thought to be highly magnetized and cold (meaning that its thermal energy is much smaller than its magnetic and bulk kinnetic energy). Over time, this wind has blown a bubble into the ejecta of the original star explosion. A reverse shock emerges when the pulsar wind first interacts the plasma filling this bubble, creating an oblate termination surface (see Fig. \ref{} in section \ref{sec:rmhd}). Downstream of this termination surface the plasma flows into the equatorial region forming a torus and towards the poles. The latter is a result of the magnetic hoopstress and forms the jet observed perpendicular to the torus.

In this article, we will review the current status of our current understanding of the plasma flow in PWN and the particle acceleration happening within it.  The article is structured as follows: first, we will introduce the RMHD models of PWN in section \ref{sec:rmhd}. In section \ref{sec:binaries} this will be extended to binary systems, in which the pulsar wind interacts with the stellar wind from a companions star. We then proceed to discuss the particle acceleration in PWN. In section \ref{sec:wisps} we discuss what can be learned about the particle acceleration from the dynamical structured called ``wisps'' observed in the Crab nebula. The we proceed to discuss the gamma-ray flares observed in this systems in section \ref{sec:flares}. The article concludes with a discussion of solved and unsolved problems of our understanding of PWN in section \ref{sec:discussion}.