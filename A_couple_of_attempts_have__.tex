A couple of attempts have been made recently to build a simple semi-analytical model of Crab’s inner knot \citep{YB-15,LKP-16}. The main motivation behind these attempts is based on the fact that the structure of the upstream wind is relatively simple and one of the key factors determining the knot appearance, namely the Doppler beaming, is relatively easy to predict based on the properties of relativistic transverse MHD shock.  The most important conclusion of the studies is that the observed clear separation of the knot from the pulsar can only be achieved in models with low wind magnetisation ($\sigma \ll 1$).  Indeed, suppose that the brightness peak of the knot corresponds to the point on the shock where the shocked plasma flows directly towards the observer. Then the observed angular distance between the peak and the pulsar is determined by flow deflection angle $\Delta\delta$ at the shock (see Figure~\ref{knot-separation}). The knot will not engulf the pulsar provided the half-opening angle of the Doppler beam, $\alpha_d$ is smaller than the deflection angle along the line of sight passing through the pulsar (see Figure~\ref{knot-separation}). Given the ultra-relativistic nature of the wind, both these angles depend mostly of the upstream $\sigma$ and hence the condition $\alpha_d < \Delta\delta$ translates into the condition on $\sigma$.    