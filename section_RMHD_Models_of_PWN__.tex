\subsection{Radiative predictions from particle transport models      (OP, AL, EA)}
\label{sec:radiation}

While the general non-thermal emission processes in PWN were rapidly identified, namely synchrotron radiation from non-thermal electrons and inverse Compton scattering {\cblue of ambient radiation by} the very same electrons \citep{shklovsky:1953,dombrovsky1954nature,atoyan1996}, ever increasing spatial and temporal resolution ask for more and more refined models of the PWN emission.
In their core, current emission models follow the suggestion of \cite{kennel1984}, that non-thermal electrons are accelerated at the TS and then follow streamlines in the nebula where they experience adiabatic and radiative losses.  In the optical and X-ray bands where the cooling timescale is comparable to the crossing time, this promotes a ``center-filled'' appearance and gradual steepening of the X-ray spectral index, observed especially in young PWN \citep[e.g.][]{SlaneChen2000,SlaneHelfand2004}.  
Sophisticated methods to track the distribution of non-thermal particles injected at the TS via passive tracer scalars were devised \citep{del-zanna2006,camus2009} and are now standard practice in MHD simulations of PWN.
{\cblue While acceleration at the TS appears as} a good working assumption, the details of the particle acceleration {\cblue process} are far from understood and {\cblue the large inferred acceleration efficiency appears} in conflict with the superluminal nature of the shock {\cblue in the framework of standard models}.  
This will be discussed further in section {\cblue \ref{sec:wispacc}} where the observational signatures of various acceleration sites on the shock are compared.

On the qualitative level, synthetic maps of synchrotron emission from PWN are able to reproduce a stunning amount of features: foremost the famous jet-and-torus morphology {\cblue whose} dynamical origins were previously discussed, but also finer details seen primarily in the high resolution observations of the Crab nebula.  We highlight the aforementioned sprite, the inner knot {\cred (sections XX and ZZ)} and the wisps {\cred (section YY)}.
As the details of energy injection into the PWN are intimately related to fundamental parameters of the rotating neutron star, one can attempt to constrain pulsar parameters with the morphology of the synchrotron emission in the nebula \citep[e.g.][]{BuhlerGiomi2016}.  
In this vein, in figure \ref{fig:maps}, we show synthetic emission maps of the region around the TS from two 3D simulations: one with parameters $\sigma_0=3,\alpha=10^\circ$ (left) and one with $\sigma_0=1,\alpha=45^\circ$ (right).  While the right-hand panel provides a good match to the morphology of Crab, the torus structure is entirely suppressed in the left-hand panel which might find its likeness in one of the jet-dominated sources \citep[][]{KargaltsevPavlov2008}.  This result is not surprising: as the extent of the striped zone is decreased, more streamlines are re-focussed into the jet (cf. figure 14 of \cite{PorthKomissarov2014a}).  The recent 3D simulations presented by \cite{Olmi2016} confirm this finding.  Hence the jet/torus flux ratio could yield a valuable handle on the pulsar obliqueness.  

On the quantitative level, some progress could be made recently with detailed modeling of the inner knot of Crab nebula by \cite{YuanBlandford2015} and \cite{LyutikovKomissarov2016}.  [WE CAN REMOVE THIS, I DON'T MEAN TO STEAL YOUR THUNDER]  If the knot-feature is caused by beamed emission right behind the TS, then the local magnetisation must be $\sigma<1$, otherwise the {\cblue observed properties} of the knot cannot be reproduced.  For an inclination of the Crab pulsar of $60^\circ$ with respect to the plane of the sky and an assumed obliqueness angle of $45^\circ$, the knot falls into the striped zone and low post-shock magnetisation is in fact expected.  

Moving away from the TS, the uncertainties in radiative models increase:  currently, the synthetic maps tend to over-predict the contrast between torus and ambient emission, as well as the optical and radio polarization degree.  For example, the unresolved polarization degree of the Crab nebula in the optical band is $\Pi=9.3\pm0.3\%$ \citep{OortWalraven1956}, whereas the simulations produce an over three times higher value.
Both effects are related, e.g. with decreasing emissivity in the ordered-field torus region as compared to the turbulent bulk, the average polarization will also decrease.  
This points at several potential shortcomings of the current 3D models:  
1. Their short duration does not allow the {\cblue dense filaments due to the Rayleigh-Taylor instability to develop and penetrate deep into the nebula, an effect that} would likely increase the randomization of magnetic field.  
2. The finite resolution global simulations might over-predict the dissipation of magnetic field in the bulk.
3. The prescription of {\cblue localized acceleration, occurring at the TS only, as in the KC84 original model, is possibly overly simplified and particle re-acceleration in the bulk might have to be taken into account.}
Future {\cblue higher resolution and longer duration} 3D simulations will need to address these questions.  

Due to the inherent complexity of the data, emission maps from dynamical simulations are not well suited to constrain parameters of PWN, and steps to reduce the dimensionality must be taken.  In this sense, \cite{volpi2008} fitted the spectral energy distribution (SED) of axisymmetric MHD simulations to the Crab nebula, taking into account two species of synchrotron emitting particles (radio- and optical/Xray-electrons) and realistic target photon fields for the IC emission.  
The result can be understood as a different flavour of the $\sigma$-problem:  
In order to reproduce the high-energy jet/torus morphology and due to the strong axial compression, the average field turns out to be $3-4$ times smaller than predictions from KC84 models \citep{de-JagerHarding1992}.  Hence in order to arrive at the observed synchrotron flux, the electron density must be increased accordingly. This is problematic not only as it adds to the tension on the multiplicity parameter $\kappa$, but also since it leads to an exaggerated IC emission \cite[see also the discussion in][]{AMATO_2014}.  
With their more uniform distribution of the magnetic field even for high magnetisation, 3D models might present a way out of this dilemma and research in this direction is ongoing \citep{Olmi2016}.  


An alternative technique to the advection of passive tracer scalars was followed by \cite{PorthVorster2016}: here, X-ray synchrotron electrons are treated as test-particles that experience the Lorentz force due to the MHD background fields.  This allows to directly map out the advection and diffusion of particles embedded in the PWN flow.  It was found that due to large velocity fluctuations, the transport exhibits a diffusive character and a typical diffusion coefficient of 
\begin{align}
D_{\rm Ls} \sim \frac{1}{3} v_f L_{\rm s} = 2.1 \times 10^{27}\left(\frac{v_f}{0.5c}\right)\left(\frac{L_{\rm s}}{0.42\rm Ly}\right)\ \rm cm^2~s^{-1}\, .\label{eq:DE}
\end{align}
is suggested following the turbulence with driving scale $L_{\rm s}$ -- the size of the termination shock, and a typical velocity at this scale $v_f$.  
Note that the value of $D_{\rm Ls}$ for Crab nebula becomes very similar to {\cblue the one estimated by \citep{AmatoEtAl00} and also to} that obtained in the early model due to \cite{WeinbergSilk1976} of $1.9 \times 10^{27}\rm cm^2 s^{-1}$.  
As particles are mostly following the velocity field, e.g. the  first order drift velocity in the toroidally dominated regions $\mathbf{v}_D=\mathbf{E\times B}/B^2$ just corresponds to the flow velocity in the poloidal plane $\mathbf{v}_D=\mathbf{v}_p$, the diffusive transport is independent of the particle energy, as assumed already in the models of \cite{gratton1972,Wilson1972}.  Further discussion on diffusion in PWN can be found in \citep{tang2012}. 
 With the spherically averaged MHD simulation as background, the particle transport taking into account advection, diffusion and synchrotron losses, can be fitted to X-ray observations of three young PWN confirming that diffusion makes an important contribution in PWN. In particular, the model of \cite{PorthVorster2016} yields better fits to the gradual increase of the X-ray spectral index than a traditional KC84 model.  
