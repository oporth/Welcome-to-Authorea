\section{RMHD Models of PWN}
\label{sec:rmhd}
\subsection{Simulating the plasma flow         (SK, OP, AL, BO, EA)}
PWN are customarily modeled under assumption of ideal magnetohydrodynamics (MHD).  Although the validity of the fluid approximation in PWN is not without question\footnote{For example, on the scale of the gyroradius for X-ray emitting particles, significant wisp-like substructure is observed which prompts for a more accurate description taking into account finit gyroradius effects (MHD+PIC).}, by upholding basic conservation laws a first insight into the physics can be attained.
To estimate the plasma-parameters in the PWN, it is instructive to first study a 1D description of the wind flow.  Assuming the wind predominantly streams out in spherical r-direction, $\mathbf{B}=B_r\mathbf{\hat{e}}^r+B_\phi\mathbf{\hat{e}}^\phi$, $\mathbf{v}=v_r\mathbf{\hat{e}}^r+v_\phi\mathbf{\hat{e}}^\phi$, the relevant equations describing the flow are
%
\begin{align}
  \partial_t(\Gamma \rho) + r^{-2} \partial_r(r^2\Gamma \rho v_r) &= 0 \label{eq:continuity} \\
  \partial_t(\omega \Gamma^2 v_\phi + \mathcal{S}_\phi) + r^{-3}\partial_r(r^3(\omega v_\phi v_r \Gamma^2-B_\phi B_r)) &= 0 \label{eq:mphi}\\
  \partial_t(E) + r^{-2}\partial_r(r^2 (\omega \Gamma^2 v_r + \mathcal{S}_r)) &= 0 \label{eq:energy} \\
  \partial_t(B_\phi) + r^{-1}\partial_r(r E_\theta) &= 0 \label{eq:induction} \\
  \partial_r(r^2 B_r) &= 0 \label{eq:divb}\ .
\end{align}
%
Adopting an ideal equation of state with adiabatic index $\hat{\gamma}$ we write the enthalpy-density $\omega=\rho c^2+\hat{\gamma}/(\hat{\gamma}-1) p $.  In the above equations, the wind Poynting flux is given by $\mathbf{\mathcal{S}} = \mathbf{E \times B}$.
In the limit of vanishing particle inertia $\omega\to 0$, \cite{1973ApJ...180L.133M} obtained an exact stationary solution to the above system:  In Michel's monopolar solution, field lines are confined to cones and describe a spiral with angle $\xi$ measured from the cylindrical direction $\tan \xi = r \Omega/c$.  So, moving away from the source, the magnetic field is slowly winding up.  

Let's derive some useful relations for the stationary system, setting $\partial_t=0$.
From the $\mathbf{\nabla\cdot B}=0$ constraint (\ref{eq:divb}), we immediately see that the radial field must decay as $B_r\propto r^{-2}$.    From the induction-law (\ref{eq:induction}) together with the ideal MHD electric field $E_\theta=B_\phi v_r-B_r v_\phi$ and $B_r\propto r^{-2}$, we obtain a first conservation law:
\begin{align}
  r\Omega \equiv v_\phi - v_r \frac{B_\phi}{B_r}
\end{align}
It describes the ``angular velocity of the field lines'' $\Omega$ and gives rise to the light cylinder radius $r_{\rm LC} = c/\Omega$.  Were the field-lines rigid radial wires sticking out of the spinning source, beyond the light-cylinder plasma would be forced to rotate faster than the speed of light.  This is circumvented by induction of a toroidal component $B_\phi$ and ``winding up'' of the field.  At the light-cylinder, we have $B_\phi\simeq B_r$.  
Dividing the toroidal momentum-flux (\ref{eq:mphi}) by the particle rest-mass energy flux $r^2\Gamma \rho v_r c^2$ (\ref{eq:continuity}), we obtain the conserved angular momentum flux
%
\begin{align}
  l \equiv \frac{\omega\Gamma r v_\phi}{\rho c^2}-\frac{r B_\phi}{k c^2}
\end{align}
with the ratio of matter to magnetic flux $k\equiv\rho v_r \Gamma / B_r$ that is also constant along a streamline.  For an accelerating wind close to the speed of light ($v_r\simeq c$) we now see that $B_\phi/B_r\simeq r/r_{LC}$ and the wind becomes dominated by the toroidal magnetic field.  Similarly, the wind rotation follows $v_\phi\propto r^{-1}$ and far away from the light-cylinder, the flow is well described by a purely radial velocity $v_r$ and an entirely toroidal magnetic field $B_\phi$.

Dividing the energy flux (\ref{eq:energy}) by the rest-mass energy flux, the conserved quantity
\begin{align}
  \mu \equiv \frac{\omega \Gamma}{\rho c^2} + \frac{\mathcal{S}_r}{\Gamma \rho v_r c^2} = \Gamma (\omega/\rho c^2 + \sigma)
\end{align}
is recovered.  It is clear that the Lorentz factor of the wind cannot exceed the value of $\mu$.  In the latter equation we have introduced the magnetization or $\sigma$-parameter:
\begin{align}
  \sigma\equiv \mathcal{S}_r/(\Gamma^2\rho v_r c^2)
\end{align}
which compares the Poynting flux with the kinetic energy of the wind.  In the cold limit $\omega\to \rho c^2$, we have $\mu=\Gamma(1+\sigma)$ which shows that accelerating the wind goes hand-in-hand with decreasing its magnetization.

As we shall see later, the value for $\sigma$ at the TS as inferred from 1D and 2D models is $\sigma\sim10^{-3}-10^{-2}$.  Quite in contrast, models for the pulsar magnetosphere predict highly magnetised plasma with $\sigma\sim10^3$ \citep[e.g.][and references therein]{arons2012}.  
The conversion of magnetic energy to kinetic energy in both confined and unconfined winds has been a subject of intensive research.  EXPLAIN FAST SURFACE CONVERSION EFFICIENCY?
Although claims have been made that the ideal MHD acceleration can provide the required energy conversion, \cite[e.g.][]{vlahakis2004}, it is now widely accepted that relativistic MHD flows are very inefficient accelerators \citep[e.g.][]{2009MNRAS.394.1182K,2009ApJ...699.1789T,lyubarsky2009,Lyubarsky2010} and can achieve $\sigma\approx1$ at best.  
The discrepancy between $\sigma$ obtained via MHD acceleration of the unconfined wind and the magnetization inferred from PWN models is known as the $\sigma$-Problem.  

In a stationary state, particle number conservation (\ref{eq:continuity}) yields for the plasma density

\begin{align}
  n=\kappa\ n_{\RM GJ} \left(\frac{R_{LC}}{R}\right)^2
\end{align}
where we have introduced as constants of proportionality the Goldreich-Julian density \citep{1969ApJ...157..869G} $n_{\rm GJ} = \frac{\Omega B_{\rm LC}}{2 \pi e c}$ measured at the light cylinder $R_{\RM LC}=c/\Omega$ and the plasma multiplicity parameter $\kappa$.  The multiplicity is a measure for the number of secondary electron positron pairs that is created for each seed electron in the Pulsar magnetosphere.

ELABORATE ON MULTIPLICITY, also on ions.

If we adopt typical values for the Pulsar wind: $B_{\rm TS}=1\rm m G$, $\Omega=30\rm Hz$, $R_{\rm TS}=10^8R_{\rm LC}$, the particle density at the TS becomes $n_{\rm TS} = \kappa\, 3\times 10^{-12}\rm cm^{-3}$.
For the range of typically assumed multiplicities $\kappa=10^{4}-10^{8}$, we obtain an extremely tenous plasma.

SHOCK JUMP AND Rees & Gunn model for the nebula flow.  MORE USEFUL 1D relations

\subsection{Radiative predictions, from particle transport models      (OP, AL, EA)}