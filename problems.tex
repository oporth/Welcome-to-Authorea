\section{Solved and unsolved problems      (OP, AL, BO,SK,EA,RB)}
\label{sec:discussion}

I suggest we all contribute to the lists first and polish the discussion later (SSK). 

Solved problems: 
1) Overall dynamics of inner Crab is well described by RMHD models. The origin of Jet and Torus is well understood. Wisps and knot are explained too.
2) The sigma-problem seems resolved. The magnetic dissipation at the termination shock, 3D randomisation and dissipation of magnetic field inside PWN removes the constrain imposed on the sigma of pulsar wind by 1D and 2D axisymmetric models.

Unsolved problems:
1) Is there a non-thermal particle acceleration at the termination shock? The Crab's inner knot is the brightest part of region near the shock filled with freshly supplied plasma. Its spectrum should reveal the shock physics. 
2) The signatures of magnetic disspation in PWN spectra. Are there any?  Where are the particle acceleration/magnetic reconnection sites? Can the magnetic reconnection explain the observed spectra of PWN? 
3) What do gamma-ray flares tell us about the dissipation/particle acceleration in PWN. 
Are these exceptional events with little impact on the overall evolution/emission of PWN or a high-energy tail of main dissipative processes. Where in the nebula do these flares come from? 
4) What is the origin of radio electrons? Are they supplied by the pulsar wind and if not then why the spectrum of emission does not show a discontinuity between the radio and optical parts? 
5) What is the structure of TS in the polar region where the pulsar wind has no stripes and hence a very high magnetization is expected but upstream and downstream of the TS. What is the fate of the hihgly-magnetized flow injected into the nebula in the polar region. Does the magnetic dissipation destroy it on a light-crossing time? 


