\section{Discussion}
%      (OP, AL, BO,SK,EA,RB)
\label{sec:discussion}

Over the past two decades, modelling of PWN has lead to a good understanding of the plasma flow in these systems. Predominantly studies have focused on the Crab nebula.  The global plasma flow and dynamics in this system are well described be the simulations. The origin of its jet, the torus, the inner knot and the wisps are understood. However, other features as the knots of the inner ring or the ``thin wisp'' \cite{Hester_1995} are not present in the simulations. Also, the morphology of other PWN appears to be more complex. In the Vela PWN for instance, the receeding jet is brighter that the one pointed to the observer, at odds with the expectation of Doppler beaming in plasma flow models.

Recently, an important step forward was done by simulating the Crab nebula in 3D. The magnetic dissipation at the termination shock and 3D randomisation of magnetic field leads to magnetic dissipation  inside PWN. This removes the constrain imposed on the low wind magnetization of pulsar wind by 1D and 2D axisymmetric models and is likely the solution to the longstanding ``sigma problem''. As always a deeper understanding comes with many new questions. Also, several older questions remained unanswered. We will summarize the ones we consider most important in the following: 

\begin{enumerate}

\item \textbf{Where and how does non-thermal particle acceleration take place? How does the particle spectrum evolve in the nebula, particularly along the termination shock? }

Magnetic reconnection appears to be the main process by which particles are accelerated in the nebula. It remains unclear where and how this happens. It is also unclear if shock acceleration takes place at particular latitudes of the reverse shock and what the contribution of second order Fermi acceleration within the body of the nebula is.

\item \textbf{What is the effect of the feedback of the acceleration on the plasma flow?}

As a significant fraction of the total energy goes into accelerating particles, their back reaction on the plasma should be important. To date, only test particle simulations have been performed, neglecting the feedback of the accelerated particles on the global plasma flow. The simple ideal MHD approximations will need to be extended to a more realistic plasma description.

\item \textbf{Where within the nebulae do the gamma-ray flares originate? Are these exceptional events with little impact on the overall evolution/emission of PWN or a high-energy tail of main dissipative processes?  }

Significant progress has been made on understanding the processes that can lead to explosive gamma-ray flares. However, if and where within the nebula these processes happen remains unclear. It is puzzling that no gamma-ray flares have been discovered from any other isolated PWN than the Crab nebula. It is not clear if similar processes are responsible for Crab nebula flares and the flares of the binary PSR B1259-63.

\item \textbf{What is the origin of radio electrons? Are they supplied by the pulsar wind and if not then why the spectrum of emission does not show a discontinuity between the radio and optical parts? }

The hard energy spectrum of the radio emitting electrons suggests that they are accelerated via magnetic reconnection. However, where these electrons come from remains an open question. If the electrons are provided by the wind their large number is in conflict with magnetospheric models. An alternative is that electrons are stripped from the gas and dust in the Nebula's filaments.

\item \textbf{What is the structure of termination shock in the polar region? }

In the polar region the pulsar wind has no stripes and hence a very high magnetization is expected both upstream and downstream of the termination shock. It is unclear what the fate of the highly magnetized flow injected into the nebula in this region and if its magnetic energy can be dissipated on a light-crossing time.

\item \textbf{How many electrons/positron escape into the ISM, with which energies? Can PWN potentially explain the positron excess?}

The interaction of PWN with the surrounding medium is complex, as they evolve within the remnant of their projenitors explosions. It is unclear how many electrons can escape the magnetic confinement of these systems and if their number is sufficient to explain the positron excess observed at Earth.

\item \textbf{Is an energetically significant amount of non-thermal  ions present in PWN?}

To date, no observational signature of ions has been found in PWN. However, due to their low radiation efficiency, there could still be an energetically significant number of ions present in the pulsar wind.

\end{enumerate}

The search for answers of these questions is ongoing. Fortunately, deeper observations and increasingly realistic simulations are powerful tools at our disposal in this quest.
