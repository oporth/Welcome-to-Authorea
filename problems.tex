\section{Solved and unsolved problems      (OP, AL, BO,SK,EA,RB)}
\label{sec:discussion}

I suggest we all contribute to the lists first and polish the discussion later (SSK). 

Solved problems: 
1) Overall dynamics of inner Crab is well described by RMHD models. The origin of Jet and Torus is well understood. Wisps and knot are explained too.

EA on item 1: We have explained all these in a framework in which the wind energy dependence was $\sin^2\theta$. What happens with $\sin^4\theta$? Can our simulations of the nebular dynamics distinguish between the two scenarios?

OP: I don't think we can distinguish in the dynamical simulations.  As shown by \cite{LyutikovKomissarov2016}, e.g. Figure 2, the differences in the radial and vertical extent of the shock are on the few percent level in a stationary model.  In a dynamical simulation it then becomes again harder to filter this effect out, see e.g. Figure 3 of \cite{BuhlerGiomi2016}, there is a small effect which makes the shock more equatorially elongated for $\sin^4\theta$ but I don't see a smoking gun.  

2) The sigma-problem seems resolved. The magnetic dissipation at the termination shock, 3D randomisation and dissipation of magnetic field inside PWN removes the constrain imposed on the sigma of pulsar wind by 1D and 2D axisymmetric models.

Comment by EA on item 2, including future work: while qualitatively it is true that the sigma problem seems resolved, it remains to be seen what are the conditions for magnetic dissipation to occur exactly in the right amount: what worries me is on the one hand the brightness contrast between the jets, torus and rest of the nebula, and on the other hand the fact that extrapolation of preliminary 3D results on the magnetic field strength to the nebula actual age leads to infer too low a field and hence again the problem of reproducing the broad band spectrum of the nebula, including Inverse Compton.
Clarifying this issue quantitatively has broad implications on our ability to constrain the width of the striped region (and hence the inclination angle of the pulsar) in different objects and also the pulsar multiplicity. 

OP: I would like to maintain that the sigma-problem as a well-posed MHD problem is solved, e.g. the shock sizes are now quantitatively comparable to the observations.  Quantitative agreement for the spectra and brightness contrast is another problem which is harder since it also depends on the particles.  If we are honest, we don't even have the tools to attack this problem rigorously.  But you are right, it is important to point out these problems.  

Unsolved problems:
1) Is there a non-thermal particle acceleration at the termination shock? The Crab's inner knot is the brightest part of region near the shock filled with freshly supplied plasma. Its spectrum should reveal the shock physics. 

EA: I would rephrase the question in terms of \underline{where} along the termination shock particle acceleration actually occurs. It must occur there somewhere, at least at the highest energies, otherwise I do not think one can understand the size shrinkage of the nebula with increasing frequency. 
The knot highlights the physics of a small fraction of the shock surface, and that the properties of the shock as an accelerator must change along the surface is clearly showed even only by the multiwavelength properties of the wisps (see Olmi et al 15) if these are to be interpreted in a pure MHD framework. However it certainly deserves our attention because it is the closest probe we have to the surface of a relativistic shock front and could provide us with direct insight into its physics. 

the workings of such a system 

and understanding of its spectrum could provide us with extremely useful information

an oblique one. It is consistent with pure thermalization and low sigma. sigma could be low from the beginning if we were looking at a region with a sufficiently large number of particles (remember Lyubarsky and Kirk 2003) a low sigma region

It appears to be consistent with pure thermalization of a particle population that  

2) The signatures of magnetic dissipation in PWN spectra. Are there any?  Where are the particle acceleration/magnetic reconnection sites? Can the magnetic reconnection explain the observed spectra of PWN? 

EA: I think the closest thing we have to a signature of magnetic reconnection is the very flat spectrum showed by the radio emitting particles. Indeed a picture in which particle acceleration occurs due to reconnection in a striped wind would smoothly connect the spectra at low and high energies. Only: the parameters do not seem right in Crab, at least on average. But of course except for the highest energies, only very local conditions at the shock surface are relevant.

3) What do gamma-ray flares tell us about the dissipation/particle acceleration in PWN. 
Are these exceptional events with little impact on the overall evolution/emission of PWN or a high-energy tail of main dissipative processes. Where in the nebula do these flares come from? 


4) What is the origin of radio electrons? Are they supplied by the pulsar wind and if not then why the spectrum of emission does not show a discontinuity between the radio and optical parts? 

EA: what are the alternatives if they are not supplied by the pulsar wind? Primordial or extracted from filaments? My objection to filaments is that we see the same spectrum (flat) in bow shock PWN where there are no Rayleigh-Taylor filaments. In addition in this case the particles would be only electrons: can we exclude this instance through circular polarization measurements? As for a primordial origin: in 2D this could be excluded in the absence of reacceleration, but with the more turbulent motions that develop in 3D....


5) What is the structure of TS in the polar region where the pulsar wind has no stripes and hence a very high magnetization is expected but upstream and downstream of the TS. What is the fate of the hihgly-magnetized flow injected into the nebula in the polar region. Does the magnetic dissipation destroy it on a light-crossing time? 


