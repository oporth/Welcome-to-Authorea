The shape of the knot based on the emissivity isophotes was another matter of concern \citep{YB-15}. Instead of being bowed away from the pulsar the knot looked bowed towards it, reflecting the shape of the termination shock ``shadow'' in the plane of the sky (see Figure~\ref{knot-polarisation}). In order to understand the significance of this result one has to study the role of other factors influencing the knot appearance. For example,  the effective geometric thickness of the emitting layer may have a strong impact as well as the variations of the velocity field in this layer \citep{LKP-16,YB-15}.  Figure~\ref{thickness-effect} illustrates the role of finite thickness in a model where the emissivity above the termination shock is set via an extrapolation of that at the shock surface \citep{LKP-16}. One can see that the effect is rather strong.    

