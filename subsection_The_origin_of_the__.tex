\subsection{The origin of the gamma-ray flares}

The Crab nebula flares pose severe challenges to models of particle acceleration. It is usually assumed that the flare emission originated from synchrotron radiation of freshly accelerated electron. The reason is that the high gamma-ray energy of the flares ($\approx 1$~GeV) implies the presence of electrons with multi PeV energies for typical magnetic fields of $B \approx 200 \mu$G expected in the nebula. The cooling time of these particles is $\tau_{cool} \lesssim 20$ days. The emission region can therefore not be far from the acceleration site (however, other views have also been proposed, see e.g. \cite{Bykov_2012}). Assuming the acceleration and radiation region are the same, the high luminosity of the flares poses severe constraints of the acceleration efficiency during the flares: the total isotropic fluence during the brightest flares is $\varepsilon \approx L_{\gamma,iso} \times t_{var} \approx XXX $~erg. Causality implies that the emission region has a volume $\approx (c t_{var})^3$. The magnetic energy in such a region us $\varepsilon_B \approx \frac{c^3}{8 \pi} t_{var}^3 B^2$. The radiation efficiency can be estimated as $\epsilon \equiv \varepsilon / \varepsilon_B \approx  5 / B^2$. 

It follows, that in order to have enough magnetic energy the flare region would need to have magnetic fields larger than several Gauss. The RMHD simulations discussed in section show that such fields are unrealistic anywhere within the body of the nebula.  This contradiction shows that flare emission cannot be isotropic and that relativistic beaming is likely playing an important role. However, it is not easy to explain highly relativistic plasma motion in the nebula. Observations show that only mildly relativistic speeds of $\approx$ 0.5c on the resolvable spatial scales. It therefore is very likely that the magnetic field in the emission region is significantly increased over the average nebula value. Furthermore, the field energy must be transferred very rapidly to kinnetic particle energy. This has lead to the idea of magnetoluminecence.


Recent ideas  \cite{Cerutti_2014}\cite{2016arXiv160403179Y}\cite{2015arXiv151205426Z}\cite{2016arXiv160304850N}\cite{2016arXiv160305731L}

Discuss: Where could this happen? transition to inner knot