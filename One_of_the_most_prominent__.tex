One of the most prominent and yet unexpected features of the synthetic synchrotron images of the simulated PWN was a very bright feature located right in the centre, where the image of the model’s pulsar would be if its emission were included (but it was not; \citet{ssk-lyub-03,ssk-lyub-04}).  A closer look revealed that the feature was a slightly off-centred and extended knot.  In order to identify the feature of the synthetic images with a particular feature of the numerical solution,  a detailed inspection of the data was carried out and it revealed that the emission originated close to the termination shock, where the shocked wind plasma was still flowing with  (moderately) relativistic speed towards the fiducial observer,  and for this reason was subject to significant Doppler-boosting.  This is illustrated in Figure~\ref{knot-mhd-model}. Subsequent  studies by other groups \cite{delzanna-06} and  the recent 3D simulations \citep{porth-13,porth-14} confirmed the result.    