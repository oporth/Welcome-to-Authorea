The first computer-generated models of the Crab Nebula focused on the case of weakly magnetised pulsar wind, thus adopting the conclusions of the Rees-Gunn-Kennel-Coroniti theory \citep{rees-gunn-74,kc84a,kc84b}.  To the delight of theorists, the simulations confirmed the possibility of the separation of the post-shock flow into the equatorial (torus)  and polar (jets) components, provided the wind magnetisation parameter $\sigma$ exceeded the critical value of few $\,\times\, 10^{-3}$ \citep{ssk-lyub-03,ssk-lyub-04,delzanna-04,bogovalov-05}.  This could be concluded immediately from the analysis of the velocity field and the distribution of other fluid parameters but in order to compared the results with the observations a synthetic emission map is highly desirable. Some elements required for the synchrotron emission calculations were readily provided by the numerical models. These are the magnetic field field strength and the orientation, as well as the velocity field, which is important for the relativistic beaming and Doppler effects.   The missing part, concerning the spectrum of the ultra-relativistic electrons, had to be recovered indirectly, via a crude model connecting the spectrum to the fluid pressure of the MHD solution\footnote{A more sophisticated technique was used in more recent simulations. }.  The un-shocked wind zone was not expected to make a significant contribution and hence was explicitly excluded from the emission calculations.
