Once we had understood the nature of the feature, it became clear that a counterpart must be present in the real images of the Crab Nebula, unless something is very wrong with the MHD model. The oblate geometry of the termination shock ensures that there always be a shock section where the post-shock plasma is streaming towards the observer with relativistic speed and hence its emission is Doppler-boosted.   Provided the Crab’s synchrotron electrons are indeed accelerated at the termination shock, as proposed in the Kennel-Coroniti model, this ensures the existence of a bright knot inside the area occupied in the plane of the sky by the un-shocked pulsar wind zone.  This knot must be a very bright permanent feature, located on the projected symmetry axis. Since the polar jets are also subject of relativistic beaming, the knot must be on the jet side of the pulsar (see Figure~\ref{knot-mhd-model}) and have no counterpart on the counter-jet side.      
A comparison with the high-resolution HST images revealed that such a feature is indeed present  in the Crab Nebula --
it is called the inner knot (or knot 1; \citet{hester-95}) and is located approximately 0.65 arcsec away from the Crab pulsar (see Figure~\ref{fig:knot}).  
So far it has been detected only in the optical-IR range.  The emission is strongly polarised with the electric polarisation vector aligned with the rotational axis in the plane of the sky. This is exactly what is expected as in the close vicinity of the termination shock the magnetic field should still preserve the highly regular azimuthal structure it has in the wind.  In a way, the inner knot was predicted by the simulations and then confirmed by the observations. Indeed, even if the observational discovery of the Crabs knot  preceded  the simulations, it was not immediately connected to the termination shock and its emergence in the synthetic maps was a complete surprise.      