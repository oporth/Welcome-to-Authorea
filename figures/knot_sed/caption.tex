The Crab knot spectrum was taken from \citep{Lobanov:2011,Sandberg2009,rudy2015}.\label{fig:sed}