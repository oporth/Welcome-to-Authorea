\subsection{Do the flares originate from the inner knot?} 

One feature which was of particular interest during the multi-wavelength campaigns was the inner knot of the nebula. This is the closest feature observed around the pulsar, located $\approx$0.6 arc sec south east of it (see Fig. \ref{fig:knot} ). The knot is detected in the infrared and optical bands. In radio and X-rays it has so far not be detected, the obtained upper limit on the flux are shown in Fig. \ref{fig:sed}. The inner knot is a dynamical feature, its distance to the pulsar can vary by $\approx$20\%  and its brightness by $\approx$50\% \cite{Sandberg2009}.  Its emission has a high degree of linear polarization \cite{Moran_2013}. Recent observations suggest that the polarization angle might change over time \cite{Moran_2015} . 

Its variability, compactness, high-level of polarization and proximity to the pulsar made the inner knot was one of the prime candidates for the site of origin of the gamma-ray flares.  In fact, it was proposed by \citet{komissarov2011}, that almost all of the emission above 100 MeV could come from this feature. Therefore, particularly dens observations targeting the inner knot where performed by the Keck Observatory and HST. Surprisingly, none of the knot properties was found to correlate to the gamma-ray emission \cite{rudy2015}. One example is shown in Fig \ref{fig:knotgamma},  where the gamma-ray flux is shown as a function of the distance of the knot from the pulsar. The inner knot is therefore likely not the site of origin of the gamma-ray flares. However, the observations revealed the dynamical behaviour of the inner knot with unprecedented detail. Its size was found to increase with the distance of the pulsar. At the same time its luminosity decreases with increasing distance, as shown in Fig. \ref{fig:knotcorr}.  As will be presented in detail in section \ref{sec:knot}, modelling of the knot shows that it is likely the high-latitude shock of the pulsar wind, giving us the first radiative signature of the wind particles.