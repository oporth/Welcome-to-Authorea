The knot does not seem to have much of an internal structure, with brightness gradually decreasing from the peak value in the centre to the background one at the periphery. It is elongated in the direction normal to the rotational axis and a little bit bowed away from the pulsar. The knot is clearly separated from the pulsar and the separation varies with  a $30\%$ amplitude on the time scale of few months, which is similar to the wisp production time scale.   The knot’s flux anti-correlates with the separation.      
The variability implies that the in the vicinity of the termination shock the flow of shocked plasma is not steady but varies on the time-scale of the shock light-crossing time.  In fact, such a variability, accompanied by a strong variation of the shock geometry, was  discovered in numerical simulations before the observations. In the high-resolution 2D simulations by \citet{camus-09}, it was found that this variability led naturally to emergence of wisps in synthetic synchrotron maps.  These wisps were identified with inhomogeneities created by the variable shock in the  equatorial outflow and advected with the outflow speed.  Thus, the similar time scales of the inner knot dynamics and the wisp production are easily explained by the fact that both are traced back to the termination shock variability.  Moreover, the recent 3D simulations of the Crab nebula are consistent with the observed knot’s flux-separation anticorrelation \citep{porth-14}.    