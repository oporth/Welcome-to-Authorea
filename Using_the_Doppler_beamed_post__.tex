Using the Doppler-beamed post-shock emissivity as a proxy for the knot brightness, one can estimate a number of observed  knot parameters such as its elongation, the distance from the pulsar in the units of the equatorial radius of the termination shock, and its polarisation degree.     The results are somewhat dependent on the  utilised model for the termination shock shape, but overall agree with the observational measurements quite well in the low-sigma-wind regime .  
The total flux polarisation degree is strongly effected by the relativistic aberration of light, which leads to a noticable rotation of the polarisation vector across the knot (see Figure~\ref{knot-polarisation}).  \citet{YB-15} concluded that the rotation imposes an upper limit 
of $\simeq 50\%$ on the overall polarisation degree of the knot, which is significantly lower than the observed $\simeq 60\%$ \sitep{moran-13}.  However, in the observations the integration was carried out only over a central part of the knot, thus excluding outer regions where the rotation is strong. \citep{LKP-16} have demonstrated that the polarisation decree decreases with the size of the integration area and this can explain the disagreement  between the observations and the upper limit found  in \citep{YB-15}.      
Moreover, the polarisation degree of the synthetic knot reproduced in the 3D simulation \citep{} agrees with the observed values exceptionally well. 