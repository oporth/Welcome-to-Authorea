The identification of the inner knot with the termination shock is already very interesting as it implies a unique opportunity for studying properties of highly relativistic shocks in magnetised plasma.  The discovery of gamma ray flares in the Crab Nebula added interest as the short life-time of the electrons emitting at such energies suggests that the gamma-ray emission originates from the terminations shock, provided it is the main acceleration site for the synchrotron electrons of all energies. Moreover, the beamed nature of the emission in this region ensures the domination of  the inner knot contribution to the observed 
gamma-ray flux \citep{ssk-lyut-11}.  Unfortunately, the angular resolution of gamma-ray telescopes is not sufficient to test this conclusion directly.  Under such circumstances, the only way to localise the flares is via their identification with structural variability in the radio, optical-IR and X-ray windows where the resolution of modern instruments is much higher. Such observations have been carried out but they have not led to a positive identification so far. Moreover, they seem to have ruled out the inner knot as the source of gamma-ray flares as its variability did not show any correlation with the flares \citep{Rudy-15}. On the other hand, these studies significantly  increased the available information on the knot properties, allowing to test its theoretical connection with termination shock in greater detail.   