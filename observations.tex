\section{Recent observational results of the Crab nebula}
\subsection{Gamma-ray flares (RB)}

The surprising discovery of gamma-ray flares form the Crab nebula gives us an additional view into the particle acceleration in the nebula (XXX). When observed in the High energy (HE, > 100 MeV) gamma-ray band, the nebula emission is highly variable. In this energy band one observes the high energy end of the synchrotron component, and the onset of the synchrotron Component in the Spectral Energy Distribution (SED) of the nebula, as shown in Fig. XXX. The monthly flux variation observed in this frequency band in the first 8 years of the Fermi-LAT mission are shown in Fig XX). A statistical analysis of the flux variations, show that the nebula emission in HE gamma-rays varies on all time scales that can be resolved. The flux variation from yearly to daily time scales can be described by a red noise process with an index of (XXX).

Extreme outbursts are observed approximately once per year. During these flares the photon flux above 100 MeV can increase up to a factor XXX. During the two brightest flares in April 2011 and March 2013 flux doubling time scales of approx. 6 hours were observed (XXX,XXX). The temporal structure of the flares is generally different between then, as are their spectral properties. Their energy spectra are typically well described by a power law, with photon indices ranging from XX-XXX. For the flares of April 2011 and March 2013 , a cutoff of the spectrum could be statistically resolved. The cutoff energy is approx 1 GeV. This appears to a a general feature of the flare spectra, for non of them significant emission was detected beyond this energy.

Unfortunately, the angular resolution in gamma rays is not sufficient to pinpoint the site of the emission region within the nebula. However, observations at lower frequency can resolve the nebula morphology in great detail. Therefore, extensive multi-wavelength campaigns have been in place since 2010 to find correlated emission to the gamma-ray flares at lower frequencies. 

Discuss: Gamma-ray observations and Multiwavelength campaigns

Figure: Long term light curve with MW observation time slots

Discuss: How could the gamma-ray flare be produced?

Recent ideas  \cite{2016arXiv160403179Y}\cite{2015arXiv151205426Z}\cite{2016arXiv160304850N}\cite{2016arXiv160305731L}

Discuss: Where could this happen? transition to inner knot

\subsection{The inner knot   (AR)}
\subsection{Is the inner knot the arch shock?                                      (SK,YY)}
