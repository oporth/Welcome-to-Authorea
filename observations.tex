\section{Recent observational results of the Crab nebula}
\subsection{Gamma-ray flares (RB)}
\subsubsection{Observational results}

Particle acceleration was observed in real time for the first time in the Crab nebula with the discovery of the gamma-ray flares in 2010.  The nebula's emission was expected to  be stable over time scales of years, it therefore came as a huge surprise when the Fermi-LAT and AGILE satellites observed strong HE gamma-ray flares (XXX). When observed in the High energy (HE, > 100 MeV) gamma-ray band, the nebula emission is highly variable. In this energy band one observes the high energy end of the synchrotron component, and the onset of the synchrotron Component in the Spectral Energy Distribution (SED) of the nebula, as shown in Fig. XXX. The monthly flux variation observed in this frequency band in the first 8 years of the Fermi-LAT mission are shown in Fig XX). A statistical analysis of the flux variations, show that the nebula emission in HE gamma-rays varies on all time scales that can be resolved by current instruments. The power density distribution (PDS) of the frequency of flux variation can be described by a red noise process with an index of $PDS \propto \nu^{-0.9}$  from yearly to daily time scales (XXX).

Extreme gamma-ray outbursts are observed approximately once per year. During these flares the photon flux above 100 MeV can increase up to a factor $\approx 30$. During the two brightest flares in April 2011 and March 2013 flux doubling time scales of approx. 6 hours were observed (XXX,XXX). The temporal structure of the flares is generally different between then, as are their spectral properties. Their photon energy spectra are typically well described by a power law $F \propto E^{-\gamma}$, with  wide range of photon indices ranging between  $\gamma \approx 1.2 - 3.7$. For the flares of April 2011 and March 2013 , a high energy cutoff of the spectrum could be statistically resolved (see figure XXX). The cutoff energy is approx $E_{cut} \approx 1$ GeV. This appears to a a general feature of the flare spectra, for none of them significant emission was detected beyond this energy. Interestingly, the time evolution of the cutoff could be resolved in time during the April 2011 flare. It was observed that the total energy flux of the gamma ray emission, $f_e$,  scales as a function of the cutoff energy as $f_e \propto E_{cut}^{3.4 \pm 0.9}$ (XXX).

Unfortunately, the angular resolution in gamma rays is not sufficient to pinpoint the site of the emission region within the nebula. However, observations at lower frequency can resolve the nebula morphology in great detail. Therefore, extensive multi-wavelength campaigns have been in place since 2010 to find correlated emission to the gamma-ray flares at lower frequencies. Some of the most sensitive instruments in the world participated in these multi-wavelength campaigns, as the Hubble Space Telescope, the Keck Observatory, Chandra X-ray Observatory and the Jansky Very Large Array (XXX,XXX). In total more than 1 Msec of observations were taken by these  instruments since 2010. Simultaneous observations on monthly intervals were carried out and additional observations taken in intervals of a few days during times in which the HE gamma-ray flux was high. Surprisingly, to date, these observations have not revealed any correlated emission to the gamma-ray flares (for more see section XXX). 

\subsubsection{The origin of the gamma-ray flares}

Recent ideas  \cite{2016arXiv160403179Y}\cite{2015arXiv151205426Z}\cite{2016arXiv160304850N}\cite{2016arXiv160305731L}

Discuss: Where could this happen? transition to inner knot

\subsection{The inner knot   (AR)}
\subsection{Is the inner knot the arch shock?                                      (SK,YY)}
