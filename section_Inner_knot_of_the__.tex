\section{Inner knot of the Crab Nebula(AR,RB)}
\label{sec:knot}

The discovery of the jet-torus structure of the inner Crab nebula coincided with the emergence of computer codes for relativistic MHD capable of dealing with highly relativistic magnetised flows \citep{ssk-godun99}.  This was most handy, as theoretical MHD models of the Crab Nebula allowed for reasonable ideas on the origin of the structure but could not address the problem rigorously due to the complicated nature of non-spherically symmetric flows.   The  key properties of pulsar winds giving rise to the jet-torus 
appearance of the nebula in these models  are 1) their anisotropy, with the wind power increasing towards the equatorial plane of the pulsar rotation 2) their magnetisation, with purely azimuthal magnetic fields aligned with the pulsar’s rotational axis.  The first property naturally leads to the torus component, whereas the second opens the possibility of magnetic collimation of the flow toward the polar axis. In fact, this collimation  mechanism fails for the wind itself, with possible exception for a tiny polar section, due to its highly super-magnetosonic motion. However, downstream of the wind termination shock the corresponding Mach number drops, the causal connectivity across the flow is restored and the magnetic hoop stress regains its collimating  potential.  Thus, in the MHD model the Crab’s jet is not present in the un-shocked pulsar wind but forms in the shocked part of the wind and thus already inside the nebula.   