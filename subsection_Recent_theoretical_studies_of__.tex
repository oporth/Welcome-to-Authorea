\subsection{Recent theoretical studies of the flares}

The Crab nebula flares pose severe challenges to models of particle acceleration. It is usually assumed that the flare emission originated from synchrotron radiation of freshly accelerated electron. The reason is that the high gamma-ray energy of the flares ($\approx 1$~GeV) implies the presence of electrons with multi PeV energies for typical magnetic fields of $B \approx 200 \mu$G expected in the nebula. The cooling time of these particles is $\tau_{cool} \lesssim 20$ days. The emission region can therefore not be far from the acceleration site  (however, other views have also been proposed, see e.g. \citet{Bykov_2012,2015arXiv151205426Z}). Assuming the acceleration and radiation region are the same, the high luminosity of the flares poses severe constraints on the acceleration efficiency during the flares: the total isotropic fluence during the brightest flares is $\varepsilon \approx L_{\gamma,iso} \times t_{var} \approx 4 \times 10^{40} $~erg. Causality implies that the emission region has a volume $\approx (c t_{var})^3$ if there is no strong Doppler beaming. The magnetic energy in such a region is $\varepsilon_B \approx \frac{c^3}{8 \pi} t_{var}^3 B^2$. The radiation efficiency can be estimated as $\epsilon \equiv \varepsilon / \varepsilon_B \approx  5 \textrm{G}^2 / B^2$. 

It follows that in order to have enough magnetic energy, the flare region needs to have magnetic fields larger than several Gauss. The RMHD simulations discussed in section \ref{sec:rmhd} show that such fields are unrealistic anywhere within the body of the nebula.  This contradiction shows that flare emission cannot be isotropic and that relativistic beaming is likely playing an important role. However, it is not easy to explain highly relativistic plasma motion in the nebula. Observations show that only mildly relativistic speeds of $\approx$ 0.5c on the resolvable spatial scales. It therefore is likely, that a combination of boosting and a high magnetic field in the emission region is required.

Dynamically, the field energy in this macroscopic region must be transferred very rapidly to kinetic particle energy (macroscopic compared to the kinetic scales as the plasma skin depth). The general concept of catastrophic dissipation of magnetic energy to non-thermal particles on macroscopic scales has been named \textit{magnetoluminescence} by \citet{Blandford_2014}. The process might be triggered by ideal MHD instabilities on large scales and produce regions with non-ideal conditions where rapid particle acceleration takes place. Recently, several works have studied this concept using force-free/MHD \citep{East_2015,2016arXiv160305731L,Zrake_2016} and PIC simulations \citep{2016arXiv160305731L,2016arXiv160403179Y,Nalewajko_2016}. For example, \citet{East_2015} found that, generic force-free equilibria within 3D periodic boxes that have ``free energy'' are unstable to ideal modes and release their magnetic energy over a single dynamic time scale. Here ``free energy'' means the amount of energy above the ground level as required by certain topological constraint (the total helicity of the magnetic field). PIC simulations further showed that the large scale instability forces current sheets to form self-consistently over dynamic time scales---these are the main sites of particle acceleration and electromagnetic dissipation \citep{2016arXiv160305731L,Nalewajko_2016,Yuan_2016}. Similar to plane current sheet reconnection case \citep[e.g.][]{Cerutti_2013,Cerutti_2014}, particles accelerated in the current sheets are beamed, at the same time bunched by the tearing modes, resulting in rapid variability of observed synchrotron radiation \citep{Yuan_2016}. While it is unlikely that these highly stylized force-free configurations correspond to realistic field structures in the Crab Nebula, they provide some insights into rapid dissipation of electromagnetic energy. \citet{2016arXiv160305731L} also studied similar processes in a more natural configuration, where two adjacent flux tubes with a zero total current merge and accelerate particles efficiently. It still remains to be seen if the aforementioned schemes produce all the features of the Crab flares, but the progress so far makes them promising candidates.

Independent of the actual particle acceleration process, it appears clear from the discussion above that regions of high magnetization and large relativistic motion are preferred sites for the gamma-ray flares. The intermediate latitude region just downstream of the reverse shock is one of the most promising region \citet{2016arXiv160305731L}. Interestingly, the shock in this region might correspond to the inner knot, observed very close to the pulsar, as will be discussed in the next sections. 